\documentclass{article}

\author{Pedro Henrique Limeira da Cruz}
\title{EM335 - Tecnologia Mecânica}

\usepackage[margin=0.8in]{geometry}
\usepackage{indentfirst}
\usepackage{fancyhdr}
\usepackage{tcolorbox} 
\usepackage{graphicx}
\usepackage{amsmath}
\usepackage{amssymb}
\usepackage{enumitem}
\usepackage{tabularx} % in the preamble


% Create a Todo list
\newlist{todolist}{itemize}{2}
\setlist[todolist]{label=$\square$}

\newcolumntype{Y}{>{\centering\arraybackslash}X}

% Create a new command to be used in the align environment in multiple line equations do only the last equation is numbered  
\newcommand{\n}{\nonumber \\ }
\makeatletter
\let\inserttitle\@title
\makeatother
% Set the style of the page 
\pagestyle{fancy}
\fancyhf{}
\rhead{Pedro Henrique L. da Cruz}
\lhead{\inserttitle}
\rfoot{Page \thepage}

\usepackage{hyperref}
\hypersetup{
    colorlinks=true,
    linkcolor=black,
    filecolor=magenta,
    urlcolor=cyan,
}

% Begin the Document 
\begin{document}

\maketitle
\thispagestyle{empty}

% Add the image inside a figure in as the first page
% \begin{figure}[h]
%     \begin{center}
%         \includegraphics[scale = 0.15]{/Users/pedrocruz/Documents/UNICAMP/ES101/ES101 - Robotic Arm/img/unicamp.png}
%     \end{center}
% \end{figure}

% Change to the Next page 
\newpage
\tableofcontents
\newpage

\section{Sistema de Tolerâncias e ajustes}
Em empresas muito grandes, e em projetos multidiciplinares, é normal que diferentes peças de um mesmo sistemas sejam fabricadas por empresas diferentes. Para que tais peças sejam compatíveis umas
com as outras, é muito importante a padronização do sistemas de tolerâncias e ajustes, e que todas os stakeholders as sigam. 

De forma geral, o sitema de tolerâncias e ajustes prevê 18 graus de tolerâncias-padrão, 

\subsections{Calibradores}
Em uma linha de produção, é importante sermos capazes de verificar se uma peça fabricada está dentro das especificação do projeto. Isso, entretanto, não pode demorar muito e não pode 
se tornar um bottle-neck. Para tal, é muito usado um calibrador "passa, não passa", onde um lado representa uma dimenção dentro das especificações e o outro não, se passar no lado repreesntativo da dimenção fora do padrão, a peça é rejeitada.

Ha, também, outros tipos, como:
\begin{itemize}
    \item Calibrador de bica ajustável
    \item Calibrador de rosca
\end{itemize}

Como os próprios calibradores precisam ser manufaturados, eles seguem graus de tolerânica padrão entre IT4 até IT11. Já entre a IT5 até IT 11 para peças que são acoplados.

\subsection{Série de Renard}
Quando estamos falando de sistemas de tolerâncias, estamos falando nada mais do que um grupo de valores aceitáveis a depender do tamanho nominal de uma peça VS tamanho real fabricado. 
Essa relação, entretanto, não faz sentido ser constante ou até mesmo linear (i.e o limite das dimenções aceitáveis para uma peça de 10mm é bem menor do que o limite das dimenções de peças de 1m). Para desenvolvermos um sistema que padroniza, em intervalos fáceis de calcular, mas que seguem a ideia das tolerâncias crescerem conforme as dimenções nomiais crescem, nós usamos a série de Renard, que calcula a série geométrica para que haja $n$ intervalos entre 10 e 100, como por exemplo:


\end{document}