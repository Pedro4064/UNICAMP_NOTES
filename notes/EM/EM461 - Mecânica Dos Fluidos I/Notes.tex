\documentclass{article}

\author{Pedro Henrique Limeira da Cruz}
\title{EM461 - Mecânica dos Fluidos I}

\usepackage[margin=0.8in]{geometry}
\usepackage{indentfirst}
\usepackage{fancyhdr}
\usepackage{tcolorbox}
\usepackage{graphicx}
\usepackage{amsmath}

% Create a new command to be used in the align environment in multiple line equations do only the last equation is numbered  
\newcommand{\n}{\nonumber \\ }
\makeatletter
\let\inserttitle\@title
\makeatother
% Set the style of the page 
\pagestyle{fancy}
\fancyhf{}
\rhead{Pedro Henrique L. da Cruz}
\lhead{\inserttitle}
\rfoot{Page \thepage}

% Begin the Document 
\begin{document}

    \maketitle
    \thispagestyle{empty}

    % Add the image inside a figure in as the first page
    \begin{figure}[h]
        \begin{center}
            \includegraphics[scale = 0.15]{/Users/pedrocruz/Documents/UNICAMP/ES101/ES101 - Robotic Arm/img/unicamp.png}
        \end{center}
    \end{figure}

    % Change to the Next page 
    \newpage
    \tableofcontents
    \newpage

    \section{Estática dos Fluidos}
        Antes de começarmos nossos estudos sobre a mecânica dos fluidos em movimento, iremos revisar (ou para alguns introduzir) a estática de fluidos. 

        \subsection{Equação Base - Estática de Fluidos}
            A equação mais básica da estática de fluidos é aquela que modela o campo de pressão em um fluido estático. A partir das experiências do dia-a-dia podemos verificar o principal aspecto
            sobre a pressão em uma coluna de fluido estático: 

            \begin{center}
                \textbf{A pressão Aumenta com a Profundidade}
            \end{center}

            A partir disso, e com a intenção de modelarmos matematicamente o sistema, fazemos a análise mais básica de mecânica estática, a lei de Newton. Para esse caso, entretanto, como estamos
            falando de um fluido e não de um corpo concentrado, iremos aplicar a lei de newton em um cenário diferencial, para lidarmos com pequenos volumes do fluido, como mostra a equação \ref{eq:newton_diff_base}:
            \begin{align}
                d\vec{F}_{resultante}= \vec a dm \label{eq:newton_diff_base}
            \end{align}

            A partir disso, como temos nossa lei de newton básica (mas agora aplicada para o problemas diferencial de fluidos), podemos prosseguir e identificar as forças envolvidas.

            A primeira força de campo que iremos ver e que atua nos problemas de estática de fluidos é a força oriunda da \emph{gravidade}, quando analisamos um pequeno volume diferencial do fluido, dada por:
            \begin{align}
                d\vec{F}_B = \vec{g}\rho d \forall \label{eq:forca_peso}
            \end{align}

            Onde:
            \begin{itemize}
                \item $\rho$: Massa específica. Para problemas que estaremos analisando é constante em função tanto do tempo quanto posição.
                \item $\forall$: Volume do elemento, dada em coordenadas cartesianas tal que $d\forall = dx \ dy \ dz$
                \item $g$: Aceleração da gravidade.
            \end{itemize}

            A segunda força que iremos analisar agora é a \textbf{única força de superfície} presente, tendo em vista que estamos abordado a estática de fluidos e, por conseguinte, não há a presença
            de tensão de cisalhamento, é a força de pressão de superfície $p = p(x, y, z)$, que varia conforme a posição dentro do fluido. Podemos entender essa pressão de superfície como sendo a 
            \textbf{Pressão exercida pela coluna de fluido ao redor do volume diferencial sendo estudado}.

            A partir disso, temos a Lei de Newton que governa o problema diferencial do fluido (dado pela equação \ref{eq:newton_diff_base}) e temos que as únicas forças que atua no nosso problema são a
            força de pressão $p(x, y, z)$ e a força peso do volume diferencial sob análise. Como estamos lidando com um problema de estática a somatória de todas as forças precisa ser zero, temos:


    
\end{document}