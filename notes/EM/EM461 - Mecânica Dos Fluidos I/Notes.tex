\documentclass{article}

\author{Pedro Henrique Limeira da Cruz}
\title{EM461 - Mecânica dos Fluidos I}

\usepackage[margin=0.8in]{geometry}
\usepackage{indentfirst}
\usepackage{fancyhdr}
\usepackage{tcolorbox}
\usepackage{graphicx}
\usepackage{amsmath}
\usepackage{amssymb}

% Create a new command to be used in the align environment in multiple line equations do only the last equation is numbered  
\newcommand{\n}{\nonumber \\ }
\makeatletter
\let\inserttitle\@title
\makeatother
% Set the style of the page 
\pagestyle{fancy}
\fancyhf{}
\rhead{Pedro Henrique L. da Cruz}
\lhead{\inserttitle}
\rfoot{Page \thepage}

% Begin the Document 
\begin{document}

    \maketitle
    \thispagestyle{empty}

    % Add the image inside a figure in as the first page
    \begin{figure}[h]
        \begin{center}
            \includegraphics[scale = 0.15]{/Users/pedrocruz/Documents/UNICAMP/ES101/ES101 - Robotic Arm/img/unicamp.png}
        \end{center}
    \end{figure}

    % Change to the Next page 
    \newpage
    \tableofcontents
    \newpage

    \section{Estática dos Fluidos}
        Antes de começarmos nossos estudos sobre a mecânica dos fluidos em movimento, iremos revisar (ou para alguns introduzir) a estática de fluidos. 

        \subsection{Equação Base - Estática de Fluidos}
            A equação mais básica da estática de fluidos é aquela que modela o campo de pressão em um fluido estático. A partir das experiências do dia-a-dia podemos verificar o principal aspecto
            sobre a pressão em uma coluna de fluido estático: 

            \begin{center}
                \textbf{A pressão Aumenta com a Profundidade}
            \end{center}

            A partir disso, e com a intenção de modelarmos matematicamente o sistema, fazemos a análise mais básica de mecânica estática, a lei de Newton. Para esse caso, entretanto, como estamos
            falando de um fluido e não de um corpo concentrado, iremos aplicar a lei de newton em um cenário diferencial, para lidarmos com pequenas massas (ou pequenos volumes) do fluido, como mostra a equação \ref{eq:newton_diff_base}:
            \begin{align}
                d\vec{F}_{resultante}= \vec a dm \label{eq:newton_diff_base}
            \end{align}

            A partir disso, como temos nossa lei de newton básica (mas agora aplicada para o problemas diferencial de fluidos), podemos prosseguir e identificar as forças envolvidas.

            A primeira força de campo que iremos ver e que atua nos problemas de estática de fluidos é a força oriunda da \emph{gravidade}, quando analisamos um pequeno volume diferencial do fluido, dada por:
            \begin{align}
                d\vec{F}_B = \vec{g}\rho d \forall \label{eq:forca_peso}
            \end{align}

            Onde:
            \begin{itemize}
                \item $\rho$: Massa específica. Para problemas que estaremos analisando é constante em função tanto do tempo quanto posição.
                \item $\forall$: Volume do elemento, dada em coordenadas cartesianas tal que $d\forall = dx \ dy \ dz$
                \item $g$: Aceleração da gravidade.
            \end{itemize}

            A segunda força que iremos analisar agora é a \textbf{única força de superfície} presente, tendo em vista que estamos abordado a estática de fluidos e, por conseguinte, não há a presença
            de tensão de cisalhamento, é a \textbf{força de pressão de superfície} $p = p(x, y, z)$ (um vetor com três componentes), que varia conforme a posição dentro do fluido. Podemos entender essa pressão de superfície como sendo a 
            \textbf{Pressão exercida pela coluna de fluido ao redor do volume diferencial sendo estudado}.

            A partir disso, temos a Lei de Newton que governa o problema diferencial do fluido (dado pela equação \ref{eq:newton_diff_base}) e temos que as únicas forças que atua no nosso problema são a
            força de pressão $p(x, y, z)$ e a força peso do volume diferencial sob análise. Como estamos lidando com um problema de estática a somatória de todas as forças precisa ser zero, temos:

            \begin{align}
                d\vec{F} = (-\nabla p + \rho \vec{g}) d\forall \Rightarrow \frac{d\vec{F}}{d\forall} = -\nabla p + \rho \vec g \label{eq:newton_estatica_de_fluidos}
            \end{align}

            Onde temos então a equação final \ref{eq:newton_estatica_de_fluidos}, que representa a força resultante por unidade de volume, que ao igualarmos a zero resulta em:

            \begin{align}
                -\nabla p &+ \rho \vec g = 0\label{eq:newton_resultante_estatica_de_fluidos}
            \end{align}

            Onde:
            \begin{itemize}
                \item $-\nabla p$: Força de pressão resultante por unidade de volume em um ponto. Representado pelo vetor gradiente com uma componente $x$ uma $y$ e uma $z$
                \item $\rho \vec g$: Força de campo (gravitacional) por unidade de volume em um ponto
            \end{itemize}

        \subsection{Variação de Pressão em um Fluido Estático}
            A seção anterior foi de suma importância por introduzir a modelagem amtemática básica para a estática de fluidos, com a equação \ref{eq:newton_resultante_estatica_de_fluidos}. A partir
            disso, podemos dissecar tal equação e relacionarmos o que sabemos na prática (que a pressão aumenta com a profundidade) com a modelagem.

            Para tal, precisamos primeiro considerar que estamos lidando com um \textbf{líquido incompressível}, de tal forma que $\rho=const$ e também que a gravidade é uma constante e aponta na
            direção $z$ somente. A partir disso, precisamos analisar a equação \ref{eq:newton_resultante_estatica_de_fluidos} somente no eixo $z$ , resultando em:

            \begin{align}
                -\nabla p + \rho \vec g &= 0 \n
                -\frac{\partial p}{\partial z} + \rho (-g_z) &= 0 \n
                -\frac{dp}{d z} + \rho (-g_z) &= 0 \label{eq:diff_inicial_pressao} \\
                dp &= +\rho g_z dz \n
                \int^p_{p_0} dp &= \int^z_{z_0} \rho g_z dz \n
                p - p_0 &= \rho g_z (z- z_0) \n 
                \Delta p &= \rho g h  \label{eq:final_delta_p_estatica_de_fluidos}
            \end{align}

            É importante ressaltar que, a equação \ref{eq:final_delta_p_estatica_de_fluidos} só é valida para:
            \begin{itemize}
                \item Fluido Estático
                \item A gravidade é a única força de campo
                \item O eixo $z$ é vertical voltado para cima
            \end{itemize}

            Além disso, a equação \ref{eq:diff_inicial_pressao} é importante para os casos em que o fluido não é incompressível (e por conseguinte a massa específica $\rho$ pode variar com a pressão
            $p$). Para esses casos, devemos partir nossas contas dessa equação e fazer a integral para o devido $\rho = f(p)$, como é nos casos dos gases ideais. Importante ressaltar que o sinal da
            gravidade é negativo pois o eixo $z$ é positivo para cima, \textbf{se o eixo $z$ for positivo para baixo o sinal da gravidade na equação \ref{eq:diff_inicial_pressao} seria positivo}.


        \subsection{Variação de Pressão Em um Gás Ideal}
            Como dito na parte anterior, há casos que lidaremos com fluidos compressíveis, sendo o mais notório deles os chamados \textbf{gases ideais}, onde:
            \begin{align}
                p &= \rho RT \therefore \rho = \frac{P}{RT} \label{eq:gases_ideais}
            \end{align}

            Onde: 
            \begin{itemize}
                \item $T$: Temperatura em Kelvin
                \item $P$: Pressão
                \item $\rho$: Massa Específica
                \item $R$: Constante universal dos gases (tabelada)
            \end{itemize}

            A partir disso, e da equação \ref{eq:diff_inicial_pressao}, somos capazes de determinar uma expressão para a pressão em um gás ideal.

        \newpage
            


    
\end{document}