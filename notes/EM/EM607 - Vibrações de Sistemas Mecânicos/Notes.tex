\documentclass{article}

\author{Pedro Henrique Limeira da Cruz}
\title{Trabalho Final - ES101}

\usepackage[margin=0.8in]{geometry}
\usepackage{indentfirst}
\usepackage{fancyhdr}
\usepackage{tcolorbox}
\usepackage{graphicx}
\usepackage{amsmath}
\usepackage{amssymb}
\usepackage{subcaption}

% Create a new command to be used in the align environment in multiple line equations do only the last equation is numbered  
\newcommand{\n}{\nonumber \\ }
\makeatletter
\let\inserttitle\@title
\makeatother
% Set the style of the page 
\pagestyle{fancy}
\fancyhf{}
\rhead{Pedro Henrique L. da Cruz}
\lhead{\inserttitle}
\rfoot{Page \thepage}

\usepackage{hyperref}
\hypersetup{
    colorlinks=true,
    linkcolor=black,
    filecolor=black,
    urlcolor=black,
}

% Begin the Document 
\begin{document}

\maketitle
\thispagestyle{empty}

% Add the image inside a figure in as the first page
% \begin{figure}[h]
%     \begin{center}
%         \includegraphics[scale = 0.15]{/Users/pedrocruz/Documents/UNICAMP/ES101/ES101 - Robotic Arm/img/unicamp.png}
%     \end{center}
% \end{figure}

% Change to the Next page 
\newpage

\section[Rev. Dinâmica]{Revisão de Dinâmica}

\subsection{Cinética Plana de Corpos Rígidos}

\subsubsection[Intro]{Introdução}
A cinética de corpos rígidos trata das relações entre as forças externas sobre um corpo e seu movimento resultante (que é composto pela rotação e translação). Para a abordagem a seguir, o
corpo apresenta um \textbf{CG} (Centro de Massa / Centro de gravidade), de maneira que todas as forças que atuam sobre o corpo atuam sobre ele.

No Total, para caracterizar totalmente o movimento de um corpo em um plano são necessárias 3 equações, sendo elas:
\begin{enumerate}
    \item Somatório de Forças no Eixo $X$
    \item Somatório de Forças no Eixo $Y$
    \item Somatório de Momentos Gerais
\end{enumerate}

Além disso, para analisarmos as formas e momentos supracitados, também é necessário (em primeiro lugar) a análise de \emph{DCL} (Diagrama de Corpo Livre).

\subsubsection[Eq. Gerais do Mov.]{Equações Gerais do Movimento}

Como havia sido dito anteriormente, para descrevermos por completo o movimento de um corpo em um plano é necessário 3 equações, sendo duas de forças e uma de momento. Sendo elas:
\begin{align}
    \sum \vec F = m \cdot \vec{\bar a} = \dot{\vec{G}}_{CG} \label{eq:SomaDasForcasCg} \\
    \sum \vec M_G =  I \cdot \vec{\bar\alpha} = \dot{\vec{H}}_{CG} \label{eq:soma_dos_momentos_cg}
\end{align}

Explorando mais as equações acima, temos que:
\begin{itemize}
    \item $\bar a$: Aceleração linear do centro de massa;
    \item $\alpha$: Aceleração angular do centro de massa;
    \item $\bar I$: Momento de inércia do corpo (i.e a medida de resistência à variação na velocidade de rotação devido à distribuição de massa em torno do CG)
    \item $\dot{\vec{G}}$: A variação no tempo da \emph{Quantidade de Movimento Linear} no \emph{CG}
    \item $\dot{\vec{H}}$: A variação no tempo da \emph{Quantidade de Movimento Angular} no \emph{CG}
\end{itemize}

É válido ressaltar, ainda, que a \emph{Quantidade de Movimento Linear} e a \emph{Quantidade de Movimento Angular} são grandezas vetoriais que é definida pelo produto entre a velocidade (linear e angular)
com a inércia (i.e a massa e a momento de inércia). E são de suma importância pois tem relação direta com força e momento (como visto anteriormente).


\subsubsection[Eq. Alternativa do Momento]{Equação Alternativa do Momento}

A fórmula \ref{eq:SomaDasForcasCg} modela a soma de momentos somente quando estamos analisando o sistema tomando como referencial o centro de gravidade \emph{CG}. Isso,
entretanto, nem sempre é possível, tendo em vista a complexidade que algumas topologias assumem, o que tornaria inviável fazer sua análise. Há, todavia, uma forma alternativa de
modelarmos o sistema, considerando um ponto $P$ arbitrário, a uma distância $d$ conhecida:
\begin{align}
    \sum \vec M_P = \bar I \cdot \alpha + m \bar a d \label{eq:soma_dos_momentos_ponto_p}
\end{align}

Onde $a$ é a aceleração linear no centro de gravidade.

Ao analisarmos bem a equação, podemos observar que ela nada mais é do que o momento no próprio centro de gravidade, dado pela parcela $\bar I \alpha$, somado ao momento (também
chamado de torque) que a força resultante ($\sum F = m \bar a$) gera no ponto $P$ a uma distância $d$ do CG em análise.

\subsubsection[Sis. de Corps. Interligados]{Sistemas de Corpos Interligados}

Em casos mais complexos, a principal topologia que encontramos é a de corpos extensos interligados. Um clássico exemplo disso é o problema do carro com pêndulo, onde temos um carro
(primeiro corpo) conectado a uma mola e a uma parede, que possui um pêndulo (segundo corpo) com seu  pivô de rotação localizado no \emph{CG} do carrinho.

Para problemas assim, temos a generalização das fórmulas, como sendo:

\begin{align}
    \sum \vec F = \sum m \vec{\bar a} \label{eq:soma_das_forcas_corps_interligados} \\
    \sum \vec{M}_P = \sum \bar I  \alpha + \sum m \vec{\bar a} d \label{eq:soma_dos_momentos_corps_interloigados}
\end{align}

A equação \ref{eq:soma_dos_momentos_corps_interloigados} pode, ainda, ser reescrita considerando a notação da \emph{Teoria dos Eixos Paralelos}, que é dada por:

\begin{align}
    \begin{cases}
        \sum \vec{M}_P & = I_P \alpha          \\
        I_P            & = \bar I + m \bar r^2
    \end{cases}
\end{align}

Isso é verdade pois $m \alpha \bar r^2 = m (\alpha \bar r) \bar r = m \bar a \bar r $, que podemos ver ser igual à equação \ref{eq:soma_dos_momentos_ponto_p}

\subsubsection[Aplicação]{Aplicação}

\textbf{EXEMPLO 1 - } O exemplo mais clássico para a aplicação de todos os conceitos vistos é o problema do carro com pêndulo, que é o que veremos agora:


No geral, iremos seguir os seguintes passos:
\begin{enumerate}
    \item Diagrama de corpo livre: A primeira coisa que devemos fazer em qualquer problema de dinâmica e Vibrações é desenhar o \emph{DCL}(Diagrama de corpo livre).
    \item Listagem dos Dados conhecidos: Em seguida, é de suma importância listarmos todos os dados que possuímos sobre o problema.
    \item Equações do Movimento: Nesse passo precisamos primeiramente identificar se estamos lidando com corpos interligados (e por conseguinte utilizaremos as equação
          \ref{eq:soma_das_forcas_corps_interligados} e \ref{eq:soma_dos_momentos_corps_interloigados}), ou se estamos lidando com corpos simples (e então usaremos as equações
          \ref{eq:SomaDasForcasCg}, \ref{eq:soma_dos_momentos_cg} e \ref{eq:soma_dos_momentos_ponto_p})
\end{enumerate}



\section{Introdução à Vibrações}

Vibração é o movimento repetitivo, que pode ser:
\begin{itemize}
    \item Desejado
    \item Não Desejado
\end{itemize}

Além disso, a vibração pode ser vista não somente como um movimento, mas também como a troca entre \emph{energia potencial} (e.g potencial elástico, potencial gravitacional, etc) e
\emph{energia mecânica}.

\section{Vibrações livres não Amortecidas - 1 DOF}

Primeiramente, é importante entendermos que definimos os tipos de vibrações de acordo com certas características, sendo elas:
\begin{enumerate}
    \item Graus de Liberdade: A depender do número de coordenadas que são necessários para a modelagem completa do movimento.
    \item Amortecimento: A depender se há o amortecimento, i.e a dissipação de energia de alguma forma (e.g um amortecedor de fluidos, atrito, ...). E pode ser classificado como:
          \subitem Amortecido: Quando ocorre a dissipação de energia;
          \subitem Não-Amortecido: Quando não ocorre a dissipação de energia e, por conseguinte, o movimento não para.

    \item Liberdade de Vibração: Que se refere à presença ou não de forças externas atuando no \emph{sistema} \footnote{Importante ressaltar que a definição do sistema implica na sua classificação}.
          \subitem Vibração Livre: Não há forças externas atuando no sistema;
          \subitem Vibrações Não-Livres: Há forças externas.

    \item Parâmetros Concentrados: Um sistema é dito de parâmetros concentrados quando os corpos que o integra são representados por corpos concentrados, e.g massas e molas.
\end{enumerate}


\subsection{Sistema Massa-Mola}

Um sistema massa mola, como descrito na imagem abaixo, assim como o nome sugere, é composto por:
\begin{itemize}
    \item Massa  de valor $m$
    \item Mola de rigidez $k$
    \item Gravidade \footnote{Importante ressaltar que nesse sistema estamos incluindo a terra (e por conseguinte suas ações na massa, e por isso continua sendo um problema de
              \emph{Vibração Livre})} $g$
\end{itemize}

Agora, para realizarmos a análise de vibrações desse problema, iremos seguir as seguintes etapas:
\begin{enumerate}
    \item Definição de Coordenadas
    \item Diagrama de Corpo Livre
    \item Equações de Movimento
\end{enumerate}

\subsubsection*{1º Passo - Definição de Coordenadas}

Algo muito importante de se perceber quando estamos lidando com problemas com molas (principalmente aqueles de massa-mola na vertical) é que existem dois principais pontos de
interesse de serem definidos no espaço, sendo eles:
\begin{itemize}
    \item Posição sem deformação da mola: Isso é, posição no espaço onde a mola não exerce força alguma, devido ao fato de que a força elástica $F_e$ é dada por $F_e = k \delta$
    \item Posição onde há equilíbrio entre a força elástica e as demais (principalmente força peso $F_p$)
\end{itemize}

Levando isso em consideração, o jeito mais fácil de lidar com esse dois pontos é criando dois eixos de coordenadas, cada um contendo seus respective zeros nesses ponto.
Portanto, o primeiro passo que precisamos fazer é definir uma coordenada $y$ a qual possui zero zero no ponto onde a mola
\textbf{NÃO} está deformada, \emph{i.e} ela está com seu tamanho normal.
Em seguida, definimos nosso segundo eixo de coordenadas $x$ contendo seu $x_0$ no ponto onde as forças peso e elástica se anulam.

\begin{figure}[h]
    \centering
    \includegraphics[width=.7\linewidth]{imgs/sis_mass_mola_1.png}
    \caption{Coordenadas para o Problema Massa-Mola na Vertical}
\end{figure}

Esses dois eixos de coordenadas distintos, como veremos mais para frente, facilitam bastante os nosso cálculos de equação de movimento.
A partir disso, já temos definidos todos os pontos de interesse, e podemos prosseguir para o diagrama de corpo livre \emph{DCL}.

\subsubsection*{2º Passo - Diagrama de Corpo Livre \emph{(DCL)}}

Para fazermos o DLC, precisamos primeiro considerar o bloco em uma posição arbitrária. Como já definimos os dois eixos de referência para baixo, iremos considerar o bloco em um
ponto tal que $x,y>0$, para facilitar os cálculos.

\begin{figure}[h]
    \centering
    \includegraphics[width=.3\linewidth]{imgs/sis_mass_mola_1_dcl.png}
    \caption{\emph{DCL} básico do sistema massa-mola}
\end{figure}

A partir disso, podemos fazer o DCL, que possui algumas características, que possuem o acronym (em inglês) \emph{B.R.E.A.D}:
\begin{itemize}
    \item \emph{Body} : Precisa representar o corpo que está sendo estudado, sem nenhuma outra coisa a sua volta (e.g sem paredes, pilastra, etc).
    \item \emph{Reações}: A segunda etapa de um DCL é a representação das forças de reação (como no nossa caso a força elástica)
    \item \emph{External / Body Forces}: A terceira etapa é a representação das forças externas ou do próprio corpo (como no nosso caso a força peso $F_p$).
    \item \emph{Axis}: A quarta coisa que seu $D.C.L$ precisa ter são os eixos de coordenadas.
    \item \emph{Dimension}: A útlima coisa que precisa ser colocada é, quando de interesse, as dimensões do do corpo.
\end{itemize}

É importante ressaltar que a força elástica $F_e$ é dada por $k\cdot y$, tendo em vista que o eixo de coordenadas $\vec y$ tem
seu zero (i.e sua origem) no ponto em que a mola não possui deformação. Poderíamos ficar escrevendo que a mola está com uma deformação $y_\alpha$ ou qualquer outro nome, mas para
termos menos trabalho usamos $y$ como sendo o ponto na coordenada $\vec y$, e usaremos ainda mais para frente $x$ para um ponto qualquer no eixo $\vec x$.

\subsubsection*{3º Passo - Equações de Movimento}

Como já temos o diagrama de corpo livre, podemos modelar o movimento do bloco a partir das equações de movimento que vimos em dinâmica (eqs. \ref{eq:SomaDasForcasCg},
\ref{eq:soma_dos_momentos_cg}).
Como o corpo não apresenta rotação (e nem momentos) iremos descrever somente a soma de forças do problema:

\begin{align}
    \sum \vec F & = m \cdot \vec a \n
    mg - k y    & = m \cdot \ddot y \n
    0           & = m \ddot y + ky - mg \label{eq:mov_massa_mola_temp}
\end{align}

Com isso temos a equação \ref{eq:mov_massa_mola_temp}, que descreve o movimento da massa como uma Equação Diferencial Ordinária de Segundo grau Ordinária de Segundo grau.
podemos, entretanto, simplificar essa equação considerando a relação entre a coordenada $x$, a coordenada $y$ e a deformação da mola, como mostramos abaixo:

\begin{align}
    k \Delta & = m g                                                      \\
    y        & = \Delta + x \therefore \dot y = \dot x, \ddot y = \ddot x
\end{align}

Substituindo as equações acima na equação \ref{eq:mov_massa_mola_temp} temos:

\begin{align}
    m\ddot x + kx = 0
\end{align}

E isso facilita as contas pois torna uma equação diferencial não homogênea em uma homogênea.
Além disso, podemos verificar que o \textbf{peso oscila em torno do ponto de equilíbrio estático}. Ao sabermos disso, nós temos então a possibilidade de, nos próximos exercícios,
partimos dessa última equação, considerando como nosso eixo de coordenadas tendo início no ponto de equilíbrio estático.

\subsection{Molas Equivalentes}

O exemplo da massa mola acima pode parecer irrealista, mas na realidade nós podemos simplificar diversos problemas do mundo real descrevendo certas estruturas através de sistemas de
molas, como veremos a seguir.

\begin{table}[h]
    \centering
    \begin{tabular}{|c|c|c|c|}
        \hline
        Sistema                    & Diagrama & Deformação & $k$ Equivalente \\ \hline
        Viga Engastada             &
        \begin{minipage}{.45\textwidth}
            \includegraphics[width=\linewidth]{imgs/mola_eq_1.png}
        \end{minipage}
                                   &
        $    \Delta = \frac{Pl^3}{3EI}$
                                   &
        $   k = \frac{P}{\Delta} = \frac{3EI}{l^3}$                          \\ \hline
        Viga Bi-Apoiada            &
        \begin{minipage}{.45\textwidth}
            \includegraphics[width=\linewidth]{imgs/mola_eq_2.png}
        \end{minipage}
                                   &
        $\Delta = \frac{Pl^3}{48EI}$
                                   &
        $k = \frac{P}{\Delta} = \frac{48EI}{l^3}$                            \\ \hline
        Barra em Solicitação Axial &
        \begin{minipage}{.45\textwidth}
            \centering
            \includegraphics[width=.5\linewidth]{imgs/mola_eq_3.png}
        \end{minipage}
                                   &
        $\Delta = \frac{Pl}{AE}$
                                   &
        $k = \frac{P}{\Delta} = \frac{AE}{l}$                                \\ \hline
    \end{tabular}
    \label{tab:molas_equivalentes}
    \caption{Molas Equivalentes}
\end{table}

Onde:
\begin{itemize}
    \item $P$: Força peso, $P = m\cdot g$
    \item $\Delta$: Deflexão
    \item $E$: Módulo de Elasticidade
    \item $I$: Inércia da seção transversal
\end{itemize}

\newpage

Além disso podemos, ainda, ter a associação de molas:
\begin{table}[h]
    \centering
    \begin{tabular}{|c|c|c|c|}
        \hline
        Topologia      & Diagrama & Equação & Observação                                                 \\ \hline
        Molas em Paralelos
                       &
        \begin{minipage}{.3\textwidth}
            \centering
            \includegraphics[width=.8\textwidth]{imgs/mola_eq_5.png}
        \end{minipage}
                       &
        $k = \sum_{i = 1}^n k_i$
                       &
        \begin{minipage}{.3\columnwidth}
            Temos que para molas em paralelo, todas tem o mesmo deslocamento%
        \end{minipage} \\ \hline

        Molas Em Série &
        \begin{minipage}{.3\linewidth}
            \centering
            \includegraphics[width=.2\linewidth]{imgs/mola_eq_6.png}
        \end{minipage}
                       &
        $k^{-1} = \sum_{i = 1}^n \frac{1}{k_i}$
                       &
        \begin{minipage}{0.3\columnwidth}
            Temos que para molas em série, cada uma delas sofre a mesma força.
        \end{minipage}                                \\ \hline
    \end{tabular}
    \caption{$k$ Resultante de associação de molas}
\end{table}


\subsection{Sistema Torcional}

\begin{figure}[h]
    \centering
    \includegraphics[width=.5\textwidth]{imgs/sis_torcional.png}
    \caption{Sistema Torcional e $DCL$}
\end{figure}

Para introduzirmos um sistema torcinal de vibrações, iremos considerando um sistema de um disco, com um momento de inércia $J$, conectado a um eixo engastado (fixo e sem rotação livre) em uma das suas extremidades, com uma rigidez torcional
\footnote{Similar à propriedade $k$ de molas normais, mas representa a resistência à torção da mola equivalente, que nesse caso refere-se à viga engastada}
$k_t$, como mostrado pela figura acima.

Quando lidamos com problemas relacionados com rotação, a maioria esmagadora de vezes iremos usar o momento para equacionar o movimento vibracional. Após a análise de corpo livre e
determinação da coordenada $\theta$ (necessária somente uma pois sistema apresenta somente um grau de liberdade), podemos aplicar a Lei de Newton no que tange momento e temos:

\begin{align*}
    \sum M = J \ddot \theta \Rightarrow -k_t\theta = J \ddot \theta
\end{align*}

Que pode ser escrita da seguinte forma:

\begin{align}
    \ddot\theta + \frac{k_t}{J}\theta = 0 \label{eq:som_moments_sis_torcional}
\end{align}

Podemos, então, verificar que, mesmo com uma topologia diferente, o sistema apresenta o mesmo comportamento (e mesma modelagem) do sistema massa mola, com a frequência natura sendo
$\omega_n = \sqrt{k_t/J}$

\section{Vibrações Livres de Sistemas de 1 \emph{DOF} com Amortecimento}

\subsection{Introdução}
Até o momento nós vimos situações onde não haviam forças dissipativas atuando sobre o sistema e, por conseguinte, o movimento de vibração continuava eternamente. Isso, entretanto, não
condiz com a realidade, levando em consideração que há inúmeros mecanismos pelos quais um sistema perde energia. Vários deles, entretanto, podem ser modelados por um
\textbf{amortecedor viscoso}, que é regido pela seguinte equação:

\begin{align}
    f_d = -c \dot x \label{eq:forca_amortecedor_viscoso}
\end{align}

Onde:
\begin{itemize}
    \item $c$: Constante do amortecedor
    \item $\dot x$: A velocidade
\end{itemize}

O exemplo mais comum disso é uma massa-mola-amortecedor, como mostrado na figura abaixo:
\begin{figure}[h]
    \centering
    \includegraphics[width=.4\textwidth]{imgs/sis_massa_mola_amortecedor.png}
    \caption{Sistema Massa-Mola-Amortecedor}
\end{figure}

Aplicando a Lei de Newton chegamos em:

\begin{align}
    \ddot x + \frac{c}{m}\dot x + \frac{k}{m}x = 0 \label{eq:massa_mola_amortecedor_base}
\end{align}

Rotineiramente, entretanto, quando temos um problema com amortecimento nós utilizamos a equação diferencial (que no nosso caso é a equação \ref{eq:massa_mola_amortecedor_base}) da seguinte
forma:
\begin{align}
    \ddot x + 2\zeta \omega_n\dot x + \omega_n^2 x = 0 \label{eq:massa_mola_amortecedor_classica}
\end{align}

Onde:
\begin{itemize}
    \item $\zeta$: \textbf{Fator de Amortecimento}, parâmetro adimencional que fornece uma medida do amortecimento do sistema (que veremos de forma mais detalhada mais para frente)
    \item $\omega_n$: Frequência Natural do sistema, dada por $\sqrt{k/m}$
\end{itemize}

\textbf{IMPORTANTE:} Em um sistema amortecido, o sistema \textbf{\emph{NÃO}} oscila com a frequência natural $\omega_n$, mas sim com uma frequência amortecida (também chamada de
\emph{damped frequency} $\omega_d$, que iremos averiguar mais para frente como é calculada).
\newpage

\subsection{Classificação de Sistemas Amortecidos}
Como vimos anteriormente, podemos modelar um sistema de 1 grau de liberdade amortecido pela equação \ref{eq:massa_mola_amortecedor_classica}, que tem como principal componente que
descreve o seu amortecimento como sendo o $\zeta$, chamado de fator de amortecimento.

Somos capazes de ver e entender a influência do fator de amortecimento ao analisarmos a equação diferencial característica \ref{eq:massa_mola_amortecedor_classica}, onde,
durante a sua resolução (para o qual supomos o sistema com uma resposta $x=Ae^{\lambda t}$), seu polinômio característico\footnote{Importante rever essa parte de Calc III ou Anal, mas o
    polinômio característico é usado para achar o $\lambda$ da exponencial que supomos ser a resposta do sistema} e suas raízes (que regem o comportamento exponencial da resposta) são dadas por:

\begin{align*}
    \lambda^2 + 2\zeta \omega_n \lambda + \omega_n^2 = 0 \Rightarrow
    \lambda_{1,2} = -\zeta \omega_n \pm \omega_n \sqrt{\zeta^2 - 1}
\end{align*}

A depender do valor de $\zeta$ temos que o sistema pode ser:
\begin{itemize}
    \item $\zeta < 1$: Sub-amortecido
    \item $\zeta=1$: Criticamente amortecido
    \item $\zeta > 1$: Super-amortecido
\end{itemize}

\subsubsection{Sistema Sub-Amortecido}
Dizemos que um sistema é sub-amortecido quando $\zeta < 1$, o que resulta na equação característica ter duas raízes imaginárias tal que:
\begin{align}
    \lambda_{1,2} = \sigma \pm  j \omega_d \Rightarrow \begin{cases}\sigma = -\zeta \omega_n \\ \omega_d = \omega_n \sqrt{1 - \zeta^2}\end{cases}\label{eq:raizes_caso_sub_amortecido}
\end{align}

Onde a parte imaginária da raiz é chamada de \emph{frequência amortecida} $\omega_d$.

Por fim, teremos como resposta do sistema:
\begin{align}
    x(t) = A_1 e^{\lambda_1 t} + A_2 e^{\lambda_2 t} = e^{-\zeta \omega_n t}(A_1e^{j\omega_dt} + A_2^{-j\omega_d t}) \label{eq:resp_sis_sub_amortecido}
\end{align}

Onde $A_{1,2}$ são constantes que somos capazes de achar a partir das duas condições iniciais.

\subsubsection{Sistema Criticamente-Amortecido}
Já para quando $\zeta=1$, nós chamamos o sistema de criticamente amortecido, e sua equação característica tem duas raízes reais iguais:
\begin{align}
    \lambda_1 = \lambda_2 = -\zeta \omega_n = -\omega_n \label{eq:raizes_caso_criticamente_amortecido}
\end{align}

Resultando em uma resposta que \textbf{não oscila} e que é descrita por:
\begin{align}
    x(t) = A_1 e^{-\omega_n t}  + A_2 e^{-\omega_n t}
\end{align}

\subsubsection{Sistema Super-Amortecido}
E para o último caso, chamamos de super-amortecido quando $\zeta > 1$, resultando em:
\begin{align}
    \lambda_{1,2} = \omega_n\left(-\zeta\pm \sqrt{\zeta^2 - 1}\right) = \frac{-1}{\tau_{1,2}}
\end{align}

Resultando em uma resposta, que também não oscila, que é descrita por:
\begin{align}
    x(t) = A_1e^{\frac{-t}{\tau_1}} + A_2e^{\frac{-t}{\tau_2}}
\end{align}

\section{Vibrações Forçadas de Sistemas de 1 \emph{DOF} sem Amortecimento - Excitação Harmônica}

\subsection{Equacionamento}
Até o momento vimos sistemas onde não haviam excitações externas (i.e não havia uma força externa ao sistema agindo sobre ele), o que resultava em uma modelagem por \emph{Equações
    Diferenciais Homogêneas}. Agora, entretanto, iremos estudar sistemas sem amortecimento que sofrem a ação de uma força periódica externa, chamada de excitação harmônica, que resultará em uma
\emph{Equação Diferencial Não Homogênea}, onde a resposta do sistema será composta por uma parte chamada de \emph{resposta forçada} e outra chamada de \emph{resposta homogênea}.

Para exemplificarmos, iremos analisar desde o começo o sistema abaixo:
\begin{figure}[h]
    \centering
    \includegraphics[width=.5\textwidth]{imgs/sis_massa_mola_harmo.png}
    \caption{Sistema Massa Mola com Excitação Harmônica}
    \label{fig:sis_massa_mola_exci_armonica}
\end{figure}


Como sempre, iremos começar por determinar o sistema de coordenadas mais aplicável nesse problema. Pela figura \ref{fig:sis_massa_mola_exci_armonica} podemos ver que a coordenada maisé
aplicávelé a \textbf{Coordenada X}, tendo como origem o \emph{ponto de equilíbrio estático do sistema}. A escolha de tal origem facilita muito nossas contas pois podemos considerar a força
elástica como sendo simplesmente $kx$ e desconsiderar a força peso (pois o $\delta$ inicial de deformação serve para anular a força peso).

Depois disso, precisamos fazer o $DLC$, como mostrado no diagrama da esquerda da figura \ref{fig:sis_massa_mola_exci_armonica}. Ressaltando novamente que a força peso é "accounted
for" pela escolha da origem do sistema como sendo o ponto de equilíbrio estático.

Agora o que nos resta é fazer a equação de movimento. Como o problema é unidirecional e não apresenta momentos, poderemos usar somente o somatório de forças em $x$ sendo igual a aceleração
do corpo, como descrito pela equação abaixo:

\begin{align*}
    \sum F_x = m \cdot \ddot x
\end{align*}

Onde as forças que atuam em $x$ são a força elástica e a força de excitação, resultando em:
\begin{align*}
    m \ddot x  = + F_0 \cos(\Omega t) - kx                 \\
    \ddot x + \frac{k}{m}x = \frac{F_0}{m} \cos (\Omega t) \\
    \ddot x + \omega_n ^2 x = f_0 \cos (\Omega t)          \\
\end{align*}

Que é uma equação Diferencial Ordinária não Homogênea, e portanto terá uma solução composta por duas partes:
\begin{align*}
    x = x_p + x_p
\end{align*}

A partir daqui o objetivo é a solução dessa EDO. Se lembrarmos bem de Calculo 3, o primeiro passo é a solução da parte homogênea, que nada mais é do que pegar a EDO, consiera-lá homogênea
(i.e desconsiderar que possui a força externa e iguala-la a zero) e resolver a EDO, supondo primeiramente que $x(t) = A e^{\lambda t}$.

Após as contas, veremos que a solução fica sendo:
\begin{align*}
    x_h(t) = B_1 \cos(\omega_n t) + B_2 \sin (\omega_n t), \begin{cases}
                                                               B_1 = A_1 + A_2 \\
                                                               B_2 = j(A_1 + A_2)
                                                           \end{cases}
\end{align*}

Para facilitarmos a visualização da resposta não forçada, podemos utilsar a lei de seno da soma para botar a solução acima em um formato de seno com um shift :
$$X \cdot sin(\omega_n t + \phi)\Rightarrow X\big[\sin(\omega_n t)\cos(\phi) + \cos(\omega_n t)\sin(\phi)\big] \therefore \begin{cases}
        B_1 = X\sin(\phi) \\
        B_2 = X\cos(\phi)
    \end{cases}$$

Igual dito anteriormente, isso facilita a visualização e análise da resposta, e a demonstração acima foi para provar matematicamente que se pode representar como sendo um seno com
frequência natural mais shift. Mas \textbf{NÃO} é necessário voltar e achar $B_{1,2}$ nem $A_{1,2}$, se acharmos $X$ e $\phi$ já descrevemos o problema.


Agora, então, só precisamos achar $x_p$, supondo uma solução no formato da entrada (logo iremos supor uma solução $x_p = C \cos(\Omega t)$), onde $C$ é uma constante qualquer. Depois
de supor essa solução e botar na EDO, iremos descobrir o valor de $C$ e então o que nos resta é jutar as duas soluções $x_h + x_p$ e utilizar os valores inicias dados para achar as
constantes $X$ e $\phi$ (da parte homogênea da resposta).

\subsection{Análise da Resposta}
Ao analisarmos a resposta, vemos que existem dois cenários importantes de serem ressaltados, sendo eles:
\begin{itemize}
    \item Fenômeno de Batimento
    \item Fenômeno de Ressonância
\end{itemize}


\begin{figure}[h]
    \begin{subfigure}{0.5\textwidth}
        \centering
        \includegraphics[width=\textwidth]{imgs/batimento.png}
        \caption{Fenômeno de Batimento}
    \end{subfigure}% Really important to have this comment here and no black line
    \begin{subfigure}{0.5\textwidth}
        \centering
        \includegraphics[width=\textwidth]{imgs/reso.png}
        \caption{Fenômeno de Ressonância}
    \end{subfigure}
    \caption{Fenômenos em Sistemas com Excitações Harmônicas}
\end{figure}

O Fenômeno de Batimento, como ilustrado na imagem acima, ocorre quando $\Omega \approx \omega_n$, \emph{i.e}, quando a frequência da excitação externa é próxima da frequência natural
do sistema.

Já o Fenômeno de Ressonância ocorre quando $\Omega \rightarrow \omega_n$, \emph{i.e}, quando a  frequência da excitação externa tende ao mesmo valor da frequência natural do sistema.




\newpage
\section{Vibrações Forçadas de Sistemas de 1 \emph{DOF} com Amortecimento - Excitação Harmônica}
Vimos, anteriormente, a resposta de um sistema sem amortecimento para uma entrada forçada (\emph{i.e} que possui uma força externa agindo sobre o sistema). Iremos, agora, verificar a modelagem
da resposta de um sistema que também possui \textbf{amortecimento viscoso}, dada por:
\begin{align}
    m\ddot x + c\dot x + kx = F_0\sin(\Omega t) \Rightarrow \ddot x + 2 \zeta \omega_n \dot x + \omega_n^2 x = f_0 \sin(\Omega t) \label{eq:vib_forcada_amortecida_1dof}
\end{align}

Onde:
\begin{itemize}
    \item $\omega_n = \sqrt{k/m}$
    \item $\zeta = c/(2m\omega_n)$
    \item $f_0=F_0/m$
\end{itemize}

Como estamos lidando com uma \emph{equação diferencial não homogene} por ter o termo forçante. Para esse caso, temos que a resposta do sistema $x = x_p + x_h$, \emph{i.e} ela divida em solução
homogênea (também chamada de resposta não forçada do sistema) e da solução particular (também chamada de resposta forçada), que são dadas por:
\begin{align}
    x = x_h + x_p\begin{cases}x_h(t) = Ae^{-\zeta \omega_n t} \sin(\omega_d t + \phi) \\ x_p(t) = M\sin(\Omega t) + N \cos{(\Omega t)}\end{cases}
\end{align}

A partir disso, para descobrirmos os valores de $M$ e $N$ basta nós substituirmos $x_p$ na equação diferencial dada pela equação \ref{eq:vib_forcada_amortecida_1dof} resultando em:
\begin{align}
    M & = \frac{(\omega_n^2 - \Omega^2)f_0}{(\omega_n^2 - \Omega^2)^2 + (2\zeta \omega_n\Omega)^2} = \frac{(1 - r^2) f_0/\omega_n^2}{(1 - r^2)^2 + (2 \zeta r)^2} \n
    N & = \frac{-2\zeta \omega_n \Omega f_0}{(\omega_n^2 - \omega^2)^2 + (2\zeta\omega_n\Omega)^2} = \frac{-2\zeta r f_0/\omega_n^2}{(1 - r^2)^2 + (2\zeta r)^2}
\end{align}

Onde temos que $r = \Omega / \omega_n$ é chamado de razão das frequências.

A partir disso, podemos, ainda, reescrever a resposta forçada ($x_p$) como sendo:
\begin{align}
    x_p(t) = X\sin(\Omega t - \theta)
\end{align}

Tal que:
\begin{itemize}
    \item $X = \sqrt{M^2 + N^2}$
    \item $\tan{\theta} = (2\zeta r)/(1-r^2)$
\end{itemize}

Tal forma é mais intuitiva de ser utilizada para visualizarmos (sem plotar) a resposta forçada do sistema. Ela nada mais é do que uma senoide (com uma certa amplitude $X$), com a mesma
frequência $\Omega$ da força harmônica de entrada mas com um delay (\emph{i.e} um shift $\theta$).

Além disso, essa re-escrita nos ajuda a analisar algo chamado de \textbf{Fator de Amplificação}, denotado por \textbf{$MF$}, que é dado por\footnote{É importante apontar que $MF$ é
    adimensional propositalmente, pois facilita a análize. A partir disso, se houver um outro sistema o qual você queira identificar o $MF$ nós pegamos a equação que define $X$ (a amplitude da
    resposta) e tentamos isolar (em função de $X$ e outras coisas) o resto da função que dependa somente de $r$ e $\zeta$, que interferem na amplitude da oscilação e que também são adimencionais.}:
\begin{align}
    MF = \frac{Xk}{F_0} = \frac{X\omega_n^2}{f_0} = \frac{1}{\sqrt{(1 - r^2)^2 + (2\zeta r)^2}}\label{eq:fator_de_amplificacao}
\end{align}

De tal forma que nós conseguimos analisar e determinar o comportamento da amplitude de saída em relação à razão entre $\Omega/\omega_n$ e ainda, por conseguinte, o pico de resposta em frequência (dada pelo valor máximo de $MF$) que corresponde a \textbf{Resonância de um Sistema Forçado de 1DOF
    Amortecido}. Tal ponto de resonância pode ser obtido via experimentação ou ainda analiticamente, se tivermos o modelo matemático que rege nosso sistema, de tal forma que é obtido pela análise
dos pontos críticos da função $MF$, dada pelos pontos onde a sua derivada é zero. Ao aplicarmos isso no exemplo em questão temos que:
\begin{align}
    MF_{max} = \frac{1}{2\zeta\sqrt{1-\zeta^2}}, \ \ \ r_{pico} = \sqrt{1 - 2\zeta^2} < 1
\end{align}

Podemos, ainda, estudar graficamente o comportamento do Fator de Amplificação em relação a razão de frequências e coeficiente de amortecimento ao plotarmos a equação \ref{eq:fator_de_amplificacao} em função de $r$, como
mostrado na imagem abaixo:

\begin{figure}[h]
    \centering
    \includegraphics[width=.5\textwidth]{imgs/fator_ampl.png}
\end{figure}

Onde Observamos que o ponto de máxima ocorre para $r < 1 \Rightarrow \Omega < \omega_n$.
Além disso, em um cenário onde estamos analizando um \emph{Bode Plot} da resposta em frequência é importante ressaltar que o momento onde ocorre a inversão da frequência de resposta,
\emph{i.e} ocorre um shift de $90^\circ$ entre a frequência de excitação e de resposta, representa o ponto de $r=1$, ou seja, $\Omega = \omega_n$. Isso nos ajuda a identificar o sistema
quando temos somente a resposta em frequência do sistema em questão.

\newpage
\section{Desbalanceamento Rotativo}
Como um exemplo de vibração forçada com amortecimento viscoso ecxitação harmônica  é um problema de desbalanceamento rotativo, como modelado pelo sistema abaixo:

\begin{figure}[h]
    \centering
    \includegraphics[width=.5\textwidth]{imgs/desb_rotativo.png}
    \caption{Modelagem de um Desbalanceamento Rotativo Com Amortecimento Viscoso}
\end{figure}

Esse sistema nós temos a seguinte modelagem\footnote{É importante ressaltar que $m$ é a massa que está desbalanceada, então se tiver 3 massas, mas duas delas estiverem radialmente opostas
    elas estariam balanceadas e teria somente uma das massas como desbalanceada. Além disso é importante ressaltar que $M$ é a soma de TODAS as massas, menos a desbalanceada}, feita a partir
do conceito de \emph{Sistemas de Corpos Interligados}:

\begin{align}
    \sum F = \sum M \ddot x \Rightarrow M\ddot x + c \dot x + kx = me\Omega^2 \sin(\Omega t)
\end{align}

Onde teremos uma resposta senoidal para a solução particular também como sendo $X \sin{(\Omega t - \theta)}$ com a amplitude $X$ dada por:
\begin{align}
    X = \frac{(me/M) r^2}{\sqrt{(1-r^2)^2 + (2\zeta r)^2}}
\end{align}

E o ângulo de fase sendo:
\begin{align}
    \tan{(\theta)} = \frac{2\zeta r}{ 1 - r^2}
\end{align}

Assim como vimos no capítulo anterior, podemos estudar a amplitude da resposta pela análise do \emph{Fator De amplificação}, que para esse caso é dado por:
\begin{align}
    MF = \frac{X}{(me/M)} = \frac{r^2}{\sqrt{(1-r^2)^2 + (2\zeta r )^2}}
\end{align}

A partir do qual podemos ver que o ponto máximo do Fator De Amplificação ocorre quando:
\begin{align}
    r = r_{pico} = \frac{1}{\sqrt{1 - 2\zeta^2}} > 1
\end{align}

O que implica que, diferente do sistema com excitação harmônica amortecido simples visto na seção anterior, quando estamos lidando com um \textbf{desbalanceamento rotativo temos que a frequência da
    entrada forçante é maior que a frequência natural do sistema no ponto de ressonância}. O que é um indicativo de qual modelo usar quando estamos fazendo a identificação do sistema e temos a sua resposta em
frequência (como um Bode Plot).
\newpage

\section{Transmissibilidade}
Considerando o sistema descrito abaixo, nosso objeto de estudo agora será determinar a força que chega ao suporte do sistema no cenário de \textbf{Regime Permanente}.

\begin{figure}[h]
    \centering
    \includegraphics[width=.2\textwidth]{imgs/trans.png}
\end{figure}

Ao fazermos o DCL e  a soma das forças em $x$(que tem origem no ponto de eq. estático logo desconsideramos o peso) vemos que a força $F$ aplicada sobre o suporte é:
\begin{align}
    F = \sum F_x = kx + c\dot x
\end{align}


Analisando em regime permanente (e por conseguinte analisando a resposta forçada do sistema) teremos:
\begin{align}
    F = kX\sin(\Omega t - \theta) + c \Omega X \cos(\Omega t - \theta) ,   x(t) = X\sin(\Omega t -\theta)
\end{align}

Como estamos lidando com soma de cossenos e senos, podemos representar como sendo somente um seno e uma amplitude, o que facilitará nossas contas e nossas análises (principalmente de
amplitude). Reescrevendo a função acima temos:
\begin{align}
    F & = \sqrt{(kX)^2 + (x\Omega X)^2} \sin(\Omega t - \theta - \beta) \n
      & = \sqrt{(kX)^2 + (x\Omega X)^2}  X \sin(\Omega t - \sigma)
\end{align}

A partir disso, podemos analisar que a amplitude máxima (e por conseguinte a força máxima) sofrida pelo suporte do sistema é:
\begin{align}
    F_T = X\sqrt{k^2 + (c \Omega)^2}
\end{align}

Com isso, somos capazes de introduzir o conceito de \textbf{TRANSMISSIBILIDADE} $TR$, que é uma relação entre a amplitude da força máxima que chega no suporte do sistema e a amplitude da força aplicada no sistema
como um todo:
\begin{align}
    TR = \frac{F_T}{F_0} = \frac{\sqrt{1 + (2 \zeta r)^2}}{\sqrt{(1 - r^2)^2 + (2 \zeta r)^2}}
\end{align}

Podemos, por fim, analisar o impacto da constante de amortecimento $\zeta$ e da razão das frequências $r$ tal que, para o caso $r=0$ temos o caso estático.

E podemos, então, observar que quanto menos amortecido, mais força chega no suporte pois gera uma \textbf{amplificação} da força de entrada, podendo quebrar o suporte se não projetado
propriamente. Além disso, para qualquer configuração há uma frequência de excitação que resultará em uma Transmissiblidade máxima, que ocorre quando:
\begin{align}
    r_{pico} = \frac{\sqrt{-1 + \sqrt{1 + 0 \zeta ^2}}}{2 \zeta} < 1
\end{align}

Uma análise de transmissibilidade também pode ser feita para o caso de um desbalanceamento rotativo, dado por:
\begin{align}
    TR = \frac{F_T}{me\omega_n^2} = \frac{r^2\sqrt{1 + (2\zeta r)^2}}{\sqrt{(1 - r^2)^2 + (2\zeta r)^2}}
\end{align}

Que analogamente ao fator de amplificação,  possui $r_{pico} > 1$, isso é, valor máximo de transmissibilidade rotativa para frequências de rotação maiores que a frequência natural de
vibração do sistema.

\section{Oscilação do Suporte}
Até o momento vimos casos onde há excitação harmônica no formato de uma força sendo aplicada no sistema. Uma forma que também é extremamente comum de gerar vibrações em sistemas é onde
ocorre a \textbf{Oscilação} do suporte no qual o sistema está fixado. Para tais casos, modelamos o sistema como demonstrado na figura \ref{fig:osci_suporte}.

\begin{figure}[h]
    \centering
    \includegraphics[width=.5\textwidth]{imgs/osci_suporte.png}
    \caption{Modelo de um sistema com oscilação do suporte}
    \label{fig:osci_suporte}
\end{figure}


Para casos como esse, é mais fácil estabelecermos duas coordenadas distintas, mas modelarmos (como veremos abaixo) com uma cordenada que representa a diferença entre as duas (como se fosse
uma "coordenada relativa"). Ao fazermos isso temos:
\begin{align}
    m\ddot x + c \dot x + kx = y_0\sqrt{k^2 + (c\Omega)^2}\sin{(\Omega t - \beta)}, \beta = \tan^{-1}-(c\Omega)/k = -2\zeta r
\end{align}

Cujo possui, como podemos ver acima, uma amplitude da força excitante sendo $F_0 = y_0 \sqrt{k^2 + (c\Omega)^2}$. A partir do que podemos resolver a EDO e achar:

$ \\ $
\textbf{SOLUÇÃO EM REGIME:}
\begin{align}
    x(t) & = X \sin(\Omega t - \gamma) \begin{cases}
                                           X = \frac{y_0\sqrt{1 + (2\zeta r)^2}}{\sqrt{(1 - r^2)^2 + (2 \zeta r)^2}} \\
                                           \gamma = \beta + \phi                                                     \\
                                           \beta = \tan^{-1} (-c\Omega)/k = -2\zeta r                                \\
                                           \phi = \tan^{-1} 2\zeta r / (1 - r^2)
                                       \end{cases} \\
    MF   & = X / y_0
\end{align}

\textbf{FORÇA TRANSMITIDA}
\begin{align}
    F_T = y_0 k \frac{r^2 \sqrt{1 + (2\zeta r)^2}}{\sqrt{(1 - r^2)^2 + (2\zeta r)^2}} \\
    TR = F_T / y_0 k
\end{align}


\newpage
\section{Análise Modal}

\subsection{Introdução}
A análise modal é uma forma diferente de abordar o estudo dos modos de vibração para grandes sistemas e grandes máquinas, pois nos permite (como veremos mais para frente) analisar certas partes do sistema de forma mais direta e específica do que é possível para o método usado até agora.

É dividido entre o processo teórico (no qual é dividido entre análise no domínio do tempo e análise no domínio da frequência, \emph{FRF}) e o processo experimental de análise de vibrações. A partir daqui iremos revisitar sistemas conhecidos, mas dessa vez utilizando a análise modal  e RFR.

\subsubsection*{Processo de Análise Teórico}
\begin{align*}
    \overbrace{Modelo\ Espacial }^{[M], [C], [K]}\rightarrow \underbrace{Modos\ de\ Vib.}_{"Modelo\ Modal", \omega_n, \zeta, modos\ de \ vibrar} \rightarrow \overbrace{Niveis\ de\ resposta}^{Resposta\ em\ Freq.\ e\ Resposta\ ao\ impulso}
\end{align*}

\subsubsection*{Processo de Análise Experimental}
\begin{align*}
    Propriedades\ e\ caracteristicas\ da\ resposta \rightarrow modos\ de\ vib. \rightarrow modelo\ estrutura
\end{align*}


\subsection{Sistema de 1DOF sem amortecimento}
\subsubsection*{Modelo Modal:}
\begin{align*}
    m \ddot x + kx = 0; x(t) = e^{i\omega t}
\end{align*}

Substituindo na equação diferencial temos:
\begin{align*}
    m(-\omega^2 X e^{i \omega t}) + k X e^{i \omega t} = 0 \Rightarrow k-\omega^2 m= 0
\end{align*}

\begin{itemize}
    \item Uma frequência natural $\omega_n = \sqrt{k/m}$
    \item Um modo de vibrar
\end{itemize}

\subsubsection*{Análise de Resposta em Frequência:}
\begin{itemize}
    \item Excitação do tipo $f(t) = Fe^{i\Omega t}$, com uma solução do tipo $x(t) = Xe^{i\Omega t}$,  resultando em:
\end{itemize}

\begin{align}
    (k - \Omega^2m)Xe^{i\Omega t} = Fe^{i\Omega t} \Rightarrow  \underbrace{\alpha(\Omega) =\frac{X}{F} = \frac{1}{k - \Omega^2m}}_{Receptancia\ do\ Sistema}
\end{align}

\subsection{Sistema com 1DOF Amortecido}

\subsubsection*{Modelo Modal:}
Para o movimento livre temos a seguinte EDO:
\begin{align*}
    m\ddot x + x\dot x + kx = 0, x(t) = Xe^{\lambda t} \Rightarrow m\lambda^2 + c\lambda + k = 0 \therefore \begin{cases}
                                                                                                                \lambda_{1,2} = -\zeta \omega_n \pm i \omega_n \sqrt{1-\zeta^2} \\ \omega_n = \sqrt{\frac{k}{m}} \\ \zeta = \frac{c}{2\sqrt{km}}
                                                                                                            \end{cases}
\end{align*}

\subsubsection*{Análise de Resposta em Frequência}
\begin{align*}
    m\ddot x + c\dot x + kx = f(t) = Fe^{i\Omega t}
\end{align*}

Como estamos analisando a resposta em frequência, logo estamos interessado somente na resposta em regime permanente, temos que:
\begin{align*}
    x(t) = Xe^{i\Omega t} \Rightarrow (-\Omega^2m + i\Omega c + k)Xe^{i\Omega t} = Fe^{i\Omega t}
\end{align*}

Resultando na seguinte receptância:
\begin{align}
    \alpha(\Omega) = \frac{X}{F} = \frac{1}{(k - \Omega^2 m) + i \Omega c}
\end{align}

O qual podemos observar que é complexo e possui o seguinte módulo e fáse:
\begin{align}
    |\alpha(\Omega) = \frac{1}{\sqrt{(k - \Omega^2)^2 + (\Omega c)^2}}| \\
    \theta_\alpha = -tg^{-1} \left[\frac{-\Omega c}{k - \Omega^2 m}\right]
\end{align}

Tendo ainda as seguintes \textbf{formas alternativas para resposta em frequência}:
\begin{itemize}
    \item $\alpha(\Omega)$: Receptância (deslocamento/força)
    \item $Y(\Omega)$: Mobilidade\footnote{Como a mobilidade só é a derivada da receptância, podemos ver que no domínio da frequência derivar (i.e passar do deslocamento para a velocidade) basta multiplicar por $i\Omega$, análogo a como no plano $s$ de Laplace bastava multiplicar por $s$} (velocidade/força), onde $Y(\Omega) = i \Omega \alpha(\Omega)$, $|Y(\Omega)| = \Omega |\alpha(\Omega)|$ e $\theta_y = \theta_\alpha - 90^\circ$
    \item $A(\Omega)$: Inertância (aceleração/força), onde $A(\Omega) = -\Omega^2 \alpha(\omega)$
\end{itemize}

Podemos usar, também, as relações inversas para análise de resposta em frequência:
\begin{itemize}
    \item Dynamic Stiffness (força/deslocamento)
    \item Impedância Mecânica (força/velocidade)
    \item Massa Aparente (força/aceleração)
\end{itemize}

\subsection{Representação Gráfica da FRF}
Como a FRF é uma função complexa, sua representação não é trivial, sendo as formas de visualização mais comuns sendo:
\begin{itemize}
    \item Bode Plot: Que possui 2 gráficos, sendo eles Módulo x $\Omega$ e fase x $\Omega$
    \item Diagrama de Nyquist: Parte real x Parte imaginária, frequência não explicita em um eixo
    \item Genérico: também com dois gráficos, sendo eles Parte Real x $\Omega$ e Parte Imaginária x $\Omega$
\end{itemize}

\newpage
\subsection{Análise de Assíntotas da FRF}
A fim de estudarmos o comportamento da resposta em frequência de sistemas, principalmente nos casos onde a frequência de excitação é próxima da frequência natural do sistema, iremos analisar a receptância, mobilidade e inertância para um sistema de 1 DOF não amortecido:

\subsubsection*{Receptância}
Como a receptância de um sistema de 1DOF não amortecido é dado por:
\begin{align*}
    \alpha(\Omega) = \frac{1}{k - \Omega^2m}
\end{align*}

Temos os seguintes casos:
\begin{itemize}
    \item \textbf{Para Baixas Frequências ($\Omega \rightarrow 0$)}: $\alpha(\Omega) \approx 1/k$
    \item \textbf{Para Altas Frequências ($\Omega \rightarrow +\infty$)}: $\alpha(\Omega) \approx 1/(-\Omega^2 m)$
\end{itemize}

Que, ao plotarmos, tem o seguinte comportamento:

\begin{figure}[h]
    \centering
    \includegraphics[width=.5\textwidth]{imgs/receptancia.png}
    \caption{Receptância $\alpha$ em função de $\Omega$ para  sistemas com diferentes frequências naturais}
\end{figure}

\newpage
\subsubsection*{Mobilidade}
Como a mobilidade é dada por:
\begin{align*}
    y(\Omega) = \frac{i\Omega}{k - \Omega^2m}
\end{align*}

Temos:
\begin{itemize}
    \item \textbf{Para Baixas Frequências ($\Omega \rightarrow 0$)}: $y(\Omega) \approx i\Omega/k$
    \item \textbf{Para Altas Frequências ($\Omega \rightarrow +\infty$)}: $y(\Omega) \approx -i/(\Omega m)$
\end{itemize}

\begin{figure}[h]
    \centering
    \includegraphics[width=.5\textwidth]{imgs/mobilidade.png}
    \caption{Mobilidade $|y(\Omega)|$ em função de $\Omega$ para  sistemas com diferentes frequências naturais}
\end{figure}












\end{document}