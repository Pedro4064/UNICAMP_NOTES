\documentclass{article}

\author{Pedro Henrique Limeira da Cruz}
\title{Trabalho Final - ES101}

\usepackage[margin=0.8in]{geometry}
\usepackage{indentfirst}
\usepackage{fancyhdr}
\usepackage{tcolorbox}
\usepackage{graphicx}
\usepackage{amsmath}

% Create a new command to be used in the align environment in multiple line equations do only the last equation is numbered  
\newcommand{\n}{\nonumber \\ }
\makeatletter
\let\inserttitle\@title
\makeatother
% Set the style of the page 
\pagestyle{fancy}
\fancyhf{}
\rhead{Pedro Henrique L. da Cruz}
\lhead{\inserttitle}
\rfoot{Page \thepage}

% Begin the Document 
\begin{document}

    \maketitle
    \thispagestyle{empty}

    % Add the image inside a figure in as the first page
    \begin{figure}[h]
        \begin{center}
            \includegraphics[scale = 0.15]{/Users/pedrocruz/Documents/UNICAMP/ES101/ES101 - Robotic Arm/img/unicamp.png}
        \end{center}
    \end{figure}

    % Change to the Next page 
    \newpage

    \section[Rev. Dinâmica]{Revisão de Dinâmica}

        \subsection{Cinética Plana de Corpos Rígidos}

            \subsubsection[Intro]{Introdução}
                A cinética de corpos rígidos trata das relações entre as forças externas sobre um corpo e seu movimento resultante (que é composto pela rotação e translação). Para a abordagem a seguir, o
                corpo apresenta um \textbf{CG} (Centro de Massa / Centro de gravidade), de maneira que todas as forças que atuam sobre o corpo atuam sobre ele.

                No Total, para caracterizar totalmente o movimento de um corpo em um plano são necessárias 3 equações, sendo elas:
                \begin{enumerate}
                    \item Somatório de Forças no Eixo $X$
                    \item Somatório de Forças no Eixo $Y$
                    \item Somatório de Momentos Gerais
                \end{enumerate}

                Além disso, para analisarmos as formas e momentos supracitados, também é necessário (em primeiro lugar) a análise de \emph{DCL} (Diagrama de Corpo Livre).

            \subsubsection[Eq. Gerais do Mov.]{Equações Gerais do Movimento}

                Como havia sido dito anteriormente, para descrevermos por completo o movimento de um corpo em um plano é necessário 3 equações, sendo duas de forças e uma de momento. Sendo elas:
                \begin{align}
                    \sum \vec F = m \cdot \vec{\bar a} = \dot{\vec{G}}_{CG} \label{eq: soma_das_forcas_cg} \\
                    \sum \vec M_G =  I \cdot \vec{\bar\alpha} = \dot{\vec{H}}_{CG} \label{eq: soma_dos_momentos_cg}
                \end{align}

                Explorando mais as equações acima, temos que:
                \begin{itemize}
                    \item $\bar a$: Aceleração linear do centro de massa;
                    \item $\alpha$: Aceleração nagular do centro de massa;
                    \item $\bar I$: Momento de inércia do corpo (i.e a medida de resistência à variação na velocidade de rotação devido à distribuição de massa em torno do CG)
                    \item $\dot{\vec{G}}$: A variação no tempo da \emph{Quantidade de Movimento Linear} no \emph{CG}
                    \item $\dot{\vec{H}}$: A variação no tempo da \emph{Quantidade de Movimento Angular} no \emph{CG}
                \end{itemize}

                É válido ressaltar, ainda, que a \emph{Quantidade de Movimento Linear} e a \emph{Quantidade de Movimento Angular} são grandezas vetoriais que é definida pelo produto entre a velocidade (linear e angular)
                com a inércia (i.e a massa e a momento de inércia). E são de suma importância pois tem relação direta com força e momento (como visto anteriormente).


            \subsubsection[Eq. Alternativa do Momento]{Equação Alternativa do Momento}

                A fórmula \ref{eq: soma_dos_momentos_cg} modela a soma de momentos somente quando estamos analisando o sistema tomando como referencial o centro de gravidade \emph{CG}. Isso,
                entretanto, nem sempre é possível, tendo em vista a complexidade que algumas topologias assumem, o que tornaria inviável fazer sua análise. Há, todavia, uma forma alternativa de
                modelarmos o sistema, considerando um ponto $P$ arbitrário, a uma distância $d$ conhecida:
                \begin{align}
                    \sum \vec M_P = \bar I \cdot \alpha + m \bar a d \label{eq: soma_dos_momentos_ponto_p}
                \end{align}

                Onde $a$ é a aceleração linear no centro de gravidade. 
                
                Ao analisarmos bem a equação, podemos observar que ela nada mais é do que o momento no próprio centro de gravidade, dado pela parcela $\bar I \alpha$, somado ao momento (também
                chamado de torque) que a força resultante ($\sum F = m \bar a$) gera no ponto $P$ a uma distância $d$ do CG em análise.

            \subsubsection[Sis. de Corps. Interligados]{Sistemas de Corpos Interligados}
                
                Em casos mais complexos, a principal topologia que encontramos é a de corpos extensos interligados. Um clássico exemplo disso é o problema do carro com pêndulo, onde temos um carro
                (primeiro corpo) conectado a uma mola e a uma parede, que possui um pêndulo (segundo corpo) com seu  pivô de rotação localizado no \emph{CG} do carrinho.

                Para problemas assim, temos a generalização das fórmulas, como sendo:

                \begin{align}
                    \sum \vec F = \sum m \vec{\bar a} \label{eq: soma_das_forcas_corps_interligados} \\
                    \sum \vec{M}_P = \sum \bar I  \alpha + \sum m \vec{\bar a} d \label{eq: soma_dos_momentos_corps_interloigados}
                \end{align}

                A equação \ref{eq: soma_dos_momentos_corps_interloigados} pode, ainda, ser reescrita considerando a notação da \emph{Teoria dos Eixos Paralelos}, que é dada por:
                
                \begin{align}
                    \begin{cases}
                        \sum \vec{M}_P &= I_P \alpha \\ 
                        I_P &= \bar I + m \bar r^2
                    \end{cases}
            \end{align}

            Isso é verdade pois $m \alpha \bar r^2 = m (\alpha \bar r) \bar r = m \bar a \bar r $, que podemos ver ser igual à equação \ref{eq: soma_dos_momentos_ponto_p}

        \subsubsection[Aplicação]{Aplicação}

            \textbf{EXEMPLO 1 - } O exemplo mais clássico para a aplicação de todos os conceitos vistos é o problema do carro com pêndulo, que é o que veremos agora:


            No geral, iremos seguir os seguintes passos:
            \begin{enumerate}
                \item Diagrama de corpo livre: A primeira coisa que devemos fazer em qualquer problema de dinâmica e Vibrações é desenhar o \emph{DCL}(Diagrama de corpo livre).
                \item Listagem dos Dados conhecidos: Em seguida, é de suma importância listarmos todos os dados que possuímos sobre o problema.
                \item Equações do Movimento: Nesse passo precisamos primeiramente identificar se estamos lidando com corpos interligados (e por conseguinte utilizaremos as equação \ref{eq:
                soma_das_forcas_corps_interligados} e \ref{eq: soma_dos_momentos_corps_interloigados}), ou se estamos lidando com corpos simples (e então usaremos as equações \ref{eq:
                soma_das_forcas_cg}, \ref{eq: soma_dos_momentos_cg} e \ref{eq: soma_dos_momentos_ponto_p})
            \end{enumerate}



    \section{Introdução à Vibrações}

            Vibração é o movimento repetitivo, que pode ser:
            \begin{itemize}
                \item Desejado
                \item Não Desejado
            \end{itemize}

            Além disso, a vibração pode ser vista não somente como um movimento, mas também como a troca entre \emph{energia potencial} (e.g potencial elástico, potencial gravitacional, etc) e
            \emph{energia mecânica}.

    \section{Vibrações livres não Amortecidas - 1 DOF}

            Primeiramente, é importante entendermos que definimos os tipos de vibrações de acordo com certas características, sendo elas:
            \begin{enumerate}
                \item Graus de Liberdade: A depender do número de coordenadas que são necessários para a modelagem completa do movimento.
                \item Amortecimento: A depender se há o amortecimento, i.e a dissipação de energia de alguma forma (e.g um amortecedor de fluidos, atrito, ...). E pode ser classificado como:
                    \subitem Amortecido: Quando ocorre a dissipação de energia;
                    \subitem Não-Amortecido: Quando não ocorre a dissipação de energia e, por conseguinte, o movimento não para.

                \item Liberdade de Vibração: Que se refere à presença ou não de forças externas atuando no \emph{sistema} \footnote{Importante ressaltar que a definição do sistema implica na sua classificação}.
                    \subitem Vibração Livre: Não há forças externas atuando no sistema;
                    \subitem Vibrações Não-Livres: Há forças externas.

                \item Parâmetros Concentrados: Um sistema é dito de parâmetros concentrados quando os corpos que o integra são representados por corpos concentrados, e.g massas e molas.
            \end{enumerate}


            \subsection{Sistema Massa-Mola}

                Um sistema massa mola, como descrito na imagem abaixo, assim como o nome sugere, é composto por:
                \begin{itemize}
                    \item Massa  de valor $m$
                    \item Mola de rigidez $k$
                    \item Gravidade \footnote{Importante ressaltar que nesse sistema estamos incluindo a terra (e por conseguinte suas ações na massa, e por isso continua sendo um problema de
                    \emph{Vibração Livre})} $g$
                \end{itemize}

                Agora, para realizarmos a análise de vibrações desse problema, iremos seguir as seguintes etapas:
                \begin{enumerate}
                    \item Definição de Coordenadas
                    \item Diagrama de Corpo Livre
                    \item Equações de Movimento
                \end{enumerate}

                \subsubsection*{1º Passo - Definição de Coordenadas}

                    Algo muito importante de se perceber quando estamos lidando com problemas com molas (principalmente aqueles de massa-mola na vertical) é que existem dois principais pontos de
                    interesse de serem definidos no espaço, sendo eles:
                    \begin{itemize}
                        \item Posição sem deformação da mola: Isso é, posição no espaço onde a mola não exerce força alguma, devido ao fato de que a força elástica $F_e$ é dada por $F_e = k \delta$
                        \item Posição onde há equilíbrio entre a força elástica e as demais (principalmente força peso $F_p$)
                    \end{itemize}

                    Levando isso em consideração, o jeito mais fácil de lidar com esse dois pontos é criando dois eixos de coordenadas, cada um contendo seus respective zeros nesses ponto. 
                    Portanto, o primeiro passo que precisamos fazer é definir uma coordenada $y$ a qual possui zero zero no ponto onde a mola
                    \textbf{NÃO} está deformada, \emph{i.e} ela está com seu tamanho normal.
                    Em seguida, definimos nosso segundo eixo de coordenadas $x$ contendo seu $x_0$ no ponto onde as forças peso e elástica se anulam.

                    \begin{figure}[h]
                        \centering
                        \includegraphics[width=.7\linewidth]{imgs/sis_mass_mola_1.png}
                        \caption{Coordenadas para o Problema Massa-Mola na Vertical}
                    \end{figure}

                    Esses dois eixos de coordenadas distintos, como veremos mais para frente, facilitam bastante os nosso cálculos de equação de movimento.
                    A partir disso, já temos definidos todos os pontos de interesse, e podemos prosseguir para o diagrama de corpo livre \emph{DCL}.

                \subsubsection*{2º Passo - Diagrama de Corpo Livre \emph{(DCL)}}

                    Para fazermos o DLC, precisamos primeiro considerar o bloco em uma posição arbitrária. Como já definimos os dois eixos de referência para baixo, iremos considerar o bloco em um
                    ponto tal que $x,y>0$, para facilitar os cálculos.

                    \begin{figure}[h]
                        \centering
                        \includegraphics[width=.3\linewidth]{imgs/sis_mass_mola_1_dcl.png}
                        \caption{\emph{DCL} básico do sistema massa-mola}
                    \end{figure}

                    A partir disso, podemos fazer o DCL, que possui algumas características, que possuem o acronym (em inglês) \emph{B.R.E.A.D}:
                    \begin{itemize}
                        \item \emph{Body} : Precisa representar o corpo que está sendo estudado, sem nenhuma outra coisa a sua volta (e.g sem paredes, pilastra, etc).
                        \item \emph{Reações}: A segunda etapa de um DCL é a representação das forças de reação (como no nossa caso a força elástica)
                        \item \emph{External / Body Forces}: A terceira etapa é a representação das forças externas ou do próprio corpo (como no nosso caso a força peso $F_p$).
                        \item \emph{Axis}: A quarta coisa que seu $D.C.L$ precisa ter são os eixos de coordenadas.
                        \item \emph{Dimension}: A útlima coisa que precisa ser colocada é, quando de interesse, as dimensões do do corpo.
                    \end{itemize}

                    É importante ressaltar que a força elástica $F_e$ é dada por $k\cdot y$, tendo em vista que o eixo de coordenadas $\vec y$ tem
                    seu zero (i.e sua origem) no ponto em que a mola não possui deformação. Poderíamos ficar escrevendo que a mola está com uma deformação $y_\alpha$ ou qualquer outro nome, mas para
                    termos menos trabalho usamos $y$ como sendo o ponto na coordenada $\vec y$, e usaremos ainda mais para frente $x$ para um ponto qualquer no eixo $\vec x$.

                    




            
\end{document}