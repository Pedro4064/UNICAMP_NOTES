\documentclass{article}

    \author{Pedro Henrique Limeira da Cruz}
    \title{ES663 - Eletronica para Automação Industrial}
    \input{preamble}
    % Begin the Document 
    \begin{document}
    
    \maketitle
    \thispagestyle{empty}
    
    % Add the image inside a figure in as the first page
    % \begin{figure}[h]
    %     \begin{center}
    %         \includegraphics[scale = 0.15]{/Users/pedrocruz/Documents/UNICAMP/ES101/ES101 - Robotic Arm/img/unicamp.png}
    %     \end{center}
    % \end{figure}
    
    % Change to the Next page 
    \newpage
    \tableofcontents
    \newpage

    \section{Conceitos Fundamentais}
        \subsection{Introdução}
            A eletrônica de potência, ao contrário da eletrôncia aplicada, circuitos I ou microeletrônica, tem como foco a manutenção e entrega de potência elétrica (e não de sinais ou lógicas, como nas outras matérias). 
            Tendo isso em mente, a primeira coisa que iremos ver são os diferentes conceitos de \textbf{potência}, com foco especial em sistemas e potência senoidais (tendo em vista que a maioria dos sistemas de potência)
            na atualidade são senoidais trifásicos.

        \subsection{Potência e Energia}
            \subsubsection*{Potência Instantânea}
                De uma forma geral, a potência instantânea é calculada através da corrente e da tensão que circula pelo circuito de interesse em um certo instante de tempo $t$, dado pela equação abaixo:
                \begin{align}
                    p(t) = v(t)i(t)
                \end{align}


\end{document}